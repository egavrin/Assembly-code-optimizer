\documentclass[11pt, a4paper]{uva_report}

\vdate{1 maart}
\year{2010}

\institute{Universiteit van Amsterdam}
\faculty{Faculteit voor Natuurkunde, Wiskunde en Informatica}
\department{Informatica instituut}

\title{Compilerbouw}
\subtitle{Peephole optimizer}

\author{Tim van Deurzen (5742900), Koos van Strien (5783437)}

\usepackage[dutch]{babel}

\begin{document}
\maketitle

\chapter{Peephole Optimizer}

    \section{Inleiding}
        Dit document beschrijft het resultaat van het practicum voor het vak
        Compilerbouw. De opdracht was om een \emph{Peephole Optimizer} te bouwen
        de overweg kon met {\bf SimpleScalar DLX assembly} code. Hierbij konden
        verschillende optimalisaties worden uitgevoerd, te bepalen door de
        studenten.
    
    
        \subsection{Vernieuwde algoritmes}
        In dit document zal de nieuwe versie van de peephole optimizer worden
        beschouwd. Deze versie verschilt op een aantal punten van de vorige
        versie.  Sinds de vorige versie van deze optimizer, is veel aan de
        algoritmiek verbeterd. In eerste instantie waren de resultaten niet
        consistent en konden metingen niet exact worden herhaald. Dit probleem
        is opgelost door de gebruikte algoritmes opnieuw op te bouwen op een
        minder complexe wijze. 

        De nieuwe algoritmen zijn iets makkelijker te begrijpen en zijn minder
        optimistisch dan de vorige algoritmen. Hierdoor werd het mogelijk de
        bestaande fouten te verbeteren. Ook is nog een extra algoritme
        toegevoegd, namelijk het optimaliseren van `onhandige branches'. Deze
        optimalisatie zal worden uitgediept in een volgende sectie.


    \section{Implementatie}
        De optimizer is ge\"implementeerd in Python. Deze programmeertaal (of
        eigenlijk scripttaal) is gekozen, omdat er snel mee geprogrammeerd kan
        worden. Ook lenen de ingebouwde datastructuren, zoals lists en
        dictionaries, zich goed voor een programma zoals dit.

        \noindent Het programma is vrij modulair opgebouwd. Er zijn aparte klassen voor:
        \begin{itemize}
            \item Het verwerken van gebruikersinput.
            \item Het indelen in basic blocks.
            \item Iedere optimalisatie.
        \end{itemize}

        \noindent Op deze manier is het makkelijk om verschillende optimalisaties buiten
        beschouwing te houden bij het uitvoeren.

        Ten opzichte van de vorige versie van deze optimizer is het indelen in
        basic blocks verbeterd. Momenteel worden eerst alle `leaders' opgezocht
        en worden vervolgens alle operaties vanaf een leader tot, maar
        \emph{niet} inclusief, de volgende leader in een basic block gezet.

        \subsection{De optimalisaties}
            Hier worden kort de algoritmen voor de verschillende optimalisaties
            besproken. Voor de gedetaileerde werking kan de code worden gelezen.
            Om dit document kort te houden worden alleen de aangepaste
            algoritmen beschreven. Voor de oude algoritmen kan het vorige
            verslag worden bekeken.

            \subsubsection{Copy propagation}
                Copy propagation is een optimalisatie die onnodig verplaatsen
                van data optimaliseert. Bijvoorbeeld: het verplaatsen van een
                variabele $a$ naar $b$, om vervolgens direct $b$ in het geheugen
                op te slaan, terwijl $a$ onaangepast is. Deze verplaatsing is
                onnodig en kan worden weggehaald.

                Het optimaliseren gaat als volgt:

                Iedere operatie in een basic block wordt bekeken. Wanneer de
                huidige operatie g\'e\'en `move' operatie is, stapt het
                programma gelijk door naar de volgende operatie.  Is de operatie
                wel een move, dan worden stuk voor stuk alle volgende operaties
                bekeken. Als een van die volgende operaties een aanpassing maakt
                aan de registers van de huidige \emph{move-operatie} kunnen er
                twee dingen gebeuren:

                \begin{itemize}
                    \item De loop kan worden verbroken en de move operatie wordt
                    niet geoptimaliseerd.
                    \item De huidige operatie wordt zo aangepast dat deze zonder
                    de move correct blijft functioneren.
                \end{itemize}

                Als er een ``Jump and Link'' instructie voorbij komt, wordt de
                move niet geoptimaliseerd, tenzij dat al door een eerdere
                instructie wel was gebeurd. Jump en Link instructies kunnen
                worden gezien als het aanroepen van een functie. In zo'n functie
                kunnen dingen gebeuren die effect hebben op het optimaliseren.
            
            \subsubsection{Redundant loads en stores}
                Het optimaliseren van redundante \emph{load} of \emph{store}
                operaties zorgt ervoor dat het geheugen niet meer wordt
                aangesproken dan nodig is. Zo is het niet nodig een waarde te
                laden direct volgend op een operatie die deze waarde opslaat.

                Het optimaliseren gaat als volgt:

                Er wordt over alle operaties binnen het basic block gelopen. Als
                een operatie een \emph{load} of \emph{store} operatie is, dan
                worden vervolgens alle volgende operaties nader bekeken. Als er
                een identieke \emph{load} operatie wordt gevonden, dan zal deze
                worden verwijderd. En als er een \emph{load} operatie wordt
                gevonden direct na, of dichtbij, een \emph{store} operatie dan
                wordt ook deze operatie verwijderd. Wordt er een ander type
                operatie gevonden en past deze operatie een van de registers van
                de huidige \emph{load} of \emph{store} aan, dan wordt de loop
                doorbroken en kan er geen operatie worden verwijderd.

                Ook hier geldt dat als er een ``Jump and Link'' operatie langs
                komt de loop wordt doorbroken en er geen operatie wordt
                vewijderd.

            \subsubsection{Global branch optimalisatie}
                Een extra optimalisatie, die is toegevoegd aan de tweede versie
                van deze optimizer, is een globale optimalisatie voor branches.
                Er komt in de assembly af en toe de volgende situatie voor:

                Een conditionele branch naar een bepaald label, daarna een jump
                naar een ander label en vervolgens het `doel-label' van de
                eerste branch. Deze constructie kan worden ingekort door de
                branch te `inverteren' en naar het label van de jump operatie te
                springen.  Dit levert uiteindelijk hetzelfde gedrag op, want als
                de conditie niet houdt gaat het programma naar het label van de
                (oude) jump. En in het andere geval, wordt de code onder het
                label van de branch uitgevoerd.

                Deze optimalisatie loopt niet over basic blocks, maar over alle
                operaties binnen het programma. Wanneer er een branch wordt
                gevonden, worden direct de volgende twee operaties bekeken. Als
                de hierboven omschreven situatie wordt geconstateerd, dan zal de
                code worden aangepast. De brach wordt dan ge\"inverteerd en de
                jump verwijderd.

       
        \section{Resultaten}
            Helaas leveren niet alle optimalisaties ook echt een versnelling op.
            Dit komt waarschijnlijk door de opbouw van de verschillende
            programma's. Het ene algoritme laat zich beter optimaliseren dan het
            andere.

            Hieronder een kleine tabel met versnellings-, dan wel
            vertragingspercentages:
            
            Voor deze uitvoer zijn alle optimalisaties, behalve globale branch
            optimalisatie uitgevoerd.\\
            
            \begin{tabular}{l | l | l | l}
                Benchmark & cycles origineel & cycles geoptimaliseerd &
                versnelling (\%):\\
                \hline\\
                acron & 5138949 & 5127044 & 0.23 \\
                clinpack & 1522838 & 1835313 & -20.51 \\
                dhrystone & 2884781 & 2859714 & 0.87 \\
                pi & 1709987 & 1702943 & 0.41 \\
                slalom & 2863936 & 2336204 & 18.23 \\
                whet & 2835438 & 2803240 & 1.14
            \end{tabular}
            
            \pagebreak[4]
            De volgende resultaten zijn inclusief de globale branch
            optimalisatie.\\

            \begin{tabular}{l | l | l | l}
                Benchmark & cycles origineel & cycles geoptimaliseerd &
                versnelling:\\
                \hline
                acron & 5138949 & 5576533 & -8.5 \\
                clinpack & 1522838 & 1870891 & -22.8 \\
                dhrystone & 2884781 & 2824640 & 2.08 \\
                pi & 1709987 & 1702928 & 0.41 \\
                slalom & 2863936 & 2332539 & 18.55 \\
                whet & 2835438 & 2756553 & 2.78
            \end{tabular}

            \vspace{0.5cm}
            De globale branch optimalisatie geeft helaas niet overal correcte
            resultaten. De output van de benchmark `acron' geeft een aantal
            extra regels, die niet in de originele output terug te vinden zijn.
            De regels hebben wel hetzelfde formaat en lijken eigenlijk gewoon
            extra resultaten te zijn.


        \section{Conclusie}
            Niet iedere benchmark heeft baat bij het uitvoeren van peephole
            optimalisaties. Er zitten zelfs benchmarks tussen (clinpack) die er
            behoorlijk op achteruit gaan. De gekozen algoritmes zijn niet zo
            heel erg agressief, dus misschien kunnen er nog betere resultaten
            worden behaald door wat optimistischer naar de operaties te kijken. 

            Wat heel duidelijk wordt, is dat optimaliseren van assembly
            behoorlijk complex is. Het vereist een hoop inzicht in de werking
            van de code en de manier waarop de assembly wordt uitgevoerd. 

            Een volgende peephole optimizer zou waarschijnlijk op een andere
            manier worden opgebouwd. Er zou dan beter worden gekeken naar de
            voordelen van bepaalde datastructuren met betrekking tot het vinden
            van bepaalde situaties. 

            Concluderend: het is een complexe taak om een peephole optimizer te
            bouwen, terwijl de begrippen nog geleerd worden. Maar het is een
            leuke en inspirerende uitdaging, die snel resultaat geeft.

\end{document}
