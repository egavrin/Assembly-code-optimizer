\documentclass[a4paper,10pt]{article}
\usepackage{multicol}
\usepackage{listings}
\author{Koos van Strien (5783437), Tim van Deurzen ()}
\title{Verslag}

\begin{document}

\maketitle

\section{Code representatie}
De code is in verschillende lagen onder te verdelen.

\subsection{De code-file}
Het bestand waar de code zich in bevindt is het hoogste niveau van 
abstractie. Aangezien een file al een geheel is, zou hier geen representatie
van hoeven zijn binnen de optimizer. Echter, aangezien de file ook de 
``container'' is van de eerstvolgende laag, basic blocks, bestaat binnen de
optimizer de class ``blockBuilder''. Hierin bevinden zich methoden om basic
blocks te herkennen en te achterhalen wat de targets zijn waar vanaf een 
basic block naartoe gesprongen wordt.

\subsection{De basic blocks}
Binnen de file is de code onder te verdelen in basic blocks. Basic blocks
beginnen met een mogelijke jump-target en eindigen met een jump of branch
operatie. Binnen de optimizer worden deze gerepresenteerd door instanties 
van de ``basicBlock'' class. Binnen deze class bevinden zich methoden om
informatie over basic blocks te verkrijgen en om veranderingen aan te brengen
aan een basic block.

\subsection{Operations en labels}
Het laagste niveau zijn labels en operations. Binnen de optimizer zijn deze 
instructies in zes categorieen verdeeld. Samen met de categorie voor labels
kunnen de regels code in \'e\'en van de volgende categorie"en vallen:
\begin{itemize}
\item Control
\item Load
\item Store
\item Integer Arithmetic
\item Float Arithmetic
\item System
\item Label
\end{itemize}

Elke categorie heeft een eigen class, met als parent-class ``operation''. 
Op deze manier kunnen generieke methoden op hoger niveau geimplementeerd 
worden en specifiekere methoden op een dieper niveau. Naast de ``operation''
class hebben de classes voor de categorie"en Load en Store nog een 
gezamelijke parent-class, aangezien zij somige acties delen.

Binnen de ``operation''-class is een static methode die bij aanroep 
automatisch een instantie van de juiste klasse teruggeeft.

Wijzigingen aan operations worden altijd via methoden gedaan. Aangezien elke
operation door een instantiatie wordt vertegenwoordigd, is dit een vrij
intu"itieve methode.


\section{Optimalisatie}
De optimizer heeft de volgende optimalisaties ge"implementeerd:
- redundant labels
- redundant load / stores
- copy propagation
- common subexpression elimination

\subsection{Redundant labels}
De redundant labels-optimalisatie kijkt of een basic block a als enige
operatie een jump heeft (naar basic block b). In dat geval wordt geheel basic
block a verwijderd. Verwijzingen naar basic block a worden vervangen door
verwijzingen naar basic block b. Dit scheelt dan een jump-operatie.
Met deze optimalisatie hebben we de volgende versnellingen weten te 
realiseren:

\subsection{Redundant load / stores}
Redundant load / store optimalisatie heeft als doel geen waarde uit het 
geheugen te laden die zich al in een register bevindt (redundant load), en 
geen ongewijzigde waarde terug te schrijven naar het geheugen (redundant 
store). Aangezien load- en store-operaties door de writeback erg traag 
zijn, kan dit al snel uitvoertijd schelen, met name wanneer bijvoorbeeld 
binnen een loop hierdoor geheugen-operaties weggelaten kunnen worden.

((Ook staan operaties nu soms achter een load
gescheduled omdat de waarde daarvoor nog niet beschikbaar zou zijn, maar
wanneer deze load redundant blijkt te zijn is er voor niets gewacht met het
uitvoeren van de instructie, kan de instructie verplaatst worden en daarmee
tijd gewonnen worden))

Met redundant load / store hebben we de volgende versnellingen weten te
realiseren:

\subsection{Copy propagation}
Copy propagation zou ook omschreven kunnen worden als "redundant moves
elimination". Het gaat namelijk hier om de move-instructie die een register 
kopieert, maar waarbij tegen de tijd dat het nieuwe register wordt 
uitgelezen door een instructie, het oude register de oude waarde nog heeft.

De analyse om te bepalen tot waar een register zijn oude waarde behoudt, 
heet een liveness-analyse. Deze analyse wordt gestart zodra in de code een
move-operatie voorkomt. Op dat moment worden de kenmerken van de move-
operatie aan een lijst toegevoegd. Wanneer nu verderop in de code 
instructies voorkomen die registers uitlezen, wordt gekeken of deze 
registers misschien gekopieerd zijn op een plek waarna het bronregister nog 
niet overschreven is. Wanneer dat het geval is, wordt het register vervangen
door het oorspronkelijke register.

Wanneer een bronregister overschreven wordt, wordt de move-operatie weer uit de
lijst gehaald. Wanneer echter het doelregister overschreven wordt terwijl het
bronregister nog bestaat, of wanneer het doelregister nooit meer gebruikt wordt
nadat het bronregister overschreven is, kan de volledige move-operatie
verwijderd worden.

\subsection{Common subexpression elimination}
Wanneer dezelfde berekening meerdere keren uitgevoerd wordt (met dezelfde
argumenten, dus dezelfde uitkomst) heet dit een common subexpression. De 
tweede berekening kan dan vervangen worden door een move-instructie.

De versnellingen die we met deze optimalisatie konden realiseren zijn:

In beginsel lijkt dit een vrij zinloze optimalisatie, aangezien er geen 
direct resultaat is. Een move-operatie is namelijk even duur als een
berekening. Echter, het impliceert wel dat registers eerder vrij komen, wat
weer kan resulteren in nieuwe redundant stores (die met move-operaties
eenvoudig te herkennen zijn), copy propagation en vervolgens dead code
elimination.
